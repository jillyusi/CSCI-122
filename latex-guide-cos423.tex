\documentclass{article}

\usepackage{amsmath, amsfonts, amsthm, fullpage, amssymb, algpseudocode, algorithm, graphicx, mathtools}

%%%% ADD EXAMPLE WITH FLOOR AND CEILING

% Macros
\DeclareMathOperator{\BigOm}{\mathcal{O}}
\newcommand{\BigO}[1]{\BigOm\left({#1}\right)}
\DeclareMathOperator{\BigTm}{\Theta}
\newcommand{\BigTheta}[1]{\BigTm\left({#1}\right)}
\DeclareMathOperator{\BigWm}{\Omega}
\newcommand{\BigOmega}[1]{\BigWm\left({#1}\right)}
\DeclareMathOperator{\LittleOm}{\mathrm{o}}
\newcommand{\LittleOh}[1]{\LittleOm\left({#1}\right)}
\DeclareMathOperator{\LittleWm}{\omega}
\newcommand{\LittleOmega}[1]{\LittleWm\left({#1}\right)}

\newcommand{\calP}{\mathcal{P}}
\newcommand{\Z}{\mathbb{Z}}
\newcommand{\R}{\mathbb{R}}
\newcommand{\Exp}{\mathbb{E}}
\newcommand{\Q}{\mathbb{Q}}
\newcommand{\sign}{\mathrm{sign\ }}
\newcommand{\abs}{\mathrm{abs\ }}
\newcommand{\eps}{\varepsilon}
\newcommand{\zo}{\{0, 1\}}
\newcommand{\SAT}{\mathit{SAT}}
\renewcommand{\P}{\mathbf{P}}
\newcommand{\NP}{\mathbf{NP}}
\newcommand{\coNP}{\co{NP}}
\newcommand{\co}[1]{\mathbf{co#1}}
\renewcommand{\Pr}{\mathop{\mathrm{Pr}}}

\newtheorem{theorem}{Theorem}
\newtheorem{lemma}[theorem]{Lemma}
\newtheorem{invariant}[theorem]{Invariant}
\newtheorem{corollary}[theorem]{Corollary}
\newtheorem{definition}[theorem]{Definition}
\newtheorem{property}[theorem]{Property}
\newtheorem{proposition}[theorem]{Proposition}  

\newenvironment{piecewise}{\left\{\begin{array}{l@{,\ }l}}
{\end{array}\right.}

\DeclarePairedDelimiter{\ceil}{\lceil}{\rceil}
\DeclarePairedDelimiter{\floor}{\lfloor}{\rfloor}

\pagestyle{plain}

\parskip=1.5mm
\parindent=0mm

\title{A Beginner's Guide to \LaTeX}
\author{\sc written by David Xiao \\ ({\em edited by Sachin Ravi and Kevin Wayne }) \\
}

% These are some useful macros

\algnewcommand\algorithmicinput{\textbf{INPUT:}}
\algnewcommand\INPUT{\item[\algorithmicinput]}
\algnewcommand\algorithmicoutput{\textbf{OUTPUT:}}
\algnewcommand\OUTPUT{\item[\algorithmicoutput]}

\begin{document}

% This prints out the document's title, author, and date
\maketitle

\section{Introduction}
\LaTeX{} is a document markup language (created by Leslie Lamport)
for the \TeX{} typesetting system (created by Donald Knuth).
It is widely used in academia, especially by scientists, engineers,
and mathematicians.
It has emerged as the standard way to display mathematics on the Web
(e.g., MathJax, Piazza, WordPress, and Coursera).
This document is intended for people who have never used \LaTeX{} and
want a quick crash course to get started.
For a more comprehensive introduction, check out
\begin{center}
\texttt{http://ctan.tug.org/tex-archive/info/lshort/english/lshort.pdf}
\end{center}

\section{How to find and use \LaTeX?}
Most Princeton university clusters have \LaTeX{} installed.
If you prefer,
you can install it on your home system; it is easy to find via 
your favorite search engine.
I recommend using front-end programs such as \TeX Shop or \TeX works.
Alternatively, you can use a text editor such as  \texttt{emacs}
(and its \LaTeX-mode) and the command line.
For simplicity in exposition, we will assume that you are compiling
from the command line for the rest of the guide.


\section{Basic rules}
Basic \LaTeX{} is just text with typesetting commands.  Typesetting
commands are usually preceded by ``\verb1\1'', and any arguments
are usually placed inside curly braces ``\verb1{}1''.

\LaTeX{} wraps text in adjacent lines as if they were part of the same
paragraph.  To start a new paragraph, insert an extra ``return'':

\begin{tabular}{ll}
Source: & Output: \\
\begin{minipage}{.5 \linewidth}
\begin{verbatim}
This is
one paragraph.

This is another.
\end{verbatim}
\end{minipage}
&
\framebox{\begin{minipage}{.5 \linewidth}
\parskip=1.5mm
This is
one paragraph.

This is another.
\end{minipage}}
\end{tabular}

\bigskip

To get a newline without starting a new paragraph, use \verb1\\1.

To get a comment, use the percent sign \verb1%1 at the beginning of a
line. The rest of that particular line will be commented out.

\section{Creating a new document}

The most basic document has three parts:
\begin{verbatim}
\documentclass{article}

\begin{document}
This is
one paragraph.

This is another.
\end{document}
\end{verbatim}

In practice, you typically create a new document by copying an 
existing document, deleting the stuff between the begin and end document
commands, and changing some information in the header.
In COS 423, you will use the following template:

\texttt{http://www.cs.princeton.edu/courses/archive/spring18/cos423/latex/problem0-1.tex}.

\section{Compiling}
Suppose our file is named \texttt{problem0-1.tex}.  To compile it,
invoke \texttt{pdflatex problem0-1.tex} in your Unix shell.  This
will compile the file, assuming there are no errors.\footnote{One quirk
is that you may have to invoke \texttt{latex} twice if you are using labels
or references.  See Section \ref{sec:labels}.} If there are errors, you
can quit the compiler by hitting ``\texttt{x}'' and then enter.
Unfortunately, \LaTeX{} compiler errors are often not very helpful in determining
the nature of the problem but they usually identify on which line the 
error occurred.

Once it successfully compiles, you will get a file named
\texttt{mylatexfile.pdf} that you can open and view.

\section{Organization}
It is important to organize your document.

\subsection{Sectioning}
There are two sectioning commands that will be useful for you:
\verb1\section{Name of section}1 and\\
\verb1\subsection{Name of subsection}1.

\subsection{Tables}

You can put stuff into tables by using the \texttt{tabular}
environment.  For example:

\begin{tabular}{ll}
Source: & Output: \\
\begin{minipage}{.5 \linewidth}
\begin{verbatim}
\begin{tabular}{r|cl}
1st column & 2nd column & 3rd column\\
\hline
a & b & c
\end{tabular}
\end{verbatim}
\end{minipage}
&
\framebox{\begin{minipage}{.5 \linewidth}
\parskip=1.5mm
\begin{tabular}{r|cl}
1st column & 2nd column & 3rd column\\
\hline
a & b & c
\end{tabular}
\end{minipage}}
\end{tabular}

Note that the command is called \texttt{tabular} and \emph{not}
\texttt{table}.  Important points:
\begin{itemize}
\item The ``\verb1{r|cl}1'' after the tabular \verb1\begin{tabular}1
  indicate the alignment of the three columns: right, center, and
  left.  This is mandatory as it specifies the layout of the table.
  For more columns, type more alignment commands, e.g. for a table
  with 5 columns all aligned to the right, you would use
  \verb1rrrrr1. 
\item The vertical bar \verb1|1 between the \texttt{r} and \texttt{c}
indicates that a vertical line should be drawn between those columns.'
\item The \verb1&1 separates the columns in the body of the table.
\item A \verb1\\1 signifies the end of each line of the table.
\item The command \verb1\hline1 means that a horizontal line should be
  inserted.
\end{itemize}

\subsection{Lists}

You can put stuff into bulleted and enumerated lists by using the
\texttt{enumerate} and \texttt{itemize} commands, respectively.
Nesting is allowed.
For example:

\begin{tabular}{ll}
Source: & Output: \\
\begin{minipage}{.5 \linewidth}
\begin{verbatim}
\begin{enumerate}
\item This is the first enumerated item.
    \begin{itemize}
    \item This is one bullet.
    \item This is another bullet.
    \end{itemize}
\item This is the second enumerated item.
\end{enumerate}
\end{verbatim}
\end{minipage}
&
\framebox{\begin{minipage}{.5 \linewidth}
\parskip=1.5mm

\begin{enumerate}
\item This is the first enumerated item.
\begin{itemize}
\item This is one bullet.
\item This is another.
\end{itemize}

\item This is the second enumerated item.
\end{enumerate}
\end{minipage}}
\end{tabular}


\subsection{Labels and references}
\label{sec:labels}
You can refer to a specific section number by using 
the \verb1\label{labelname}1 command.  Place this right
after you start a section.  Then, you may refer to the section number
by using \verb1\ref{labelname}1.  The same technique can be used to 
refer to math equations and theorems.

\LaTeX creates and uses a bunch of auxiliary files to support
labels and references.
Be sure to invoke \texttt{latex} twice to compile a file that has labels and
references (or if thoses labels and references have changed since the
last compilation).

\section{Math}
The primary reason to use \LaTeX{} is its ability to typeset
mathematical expressions. 

\subsection{Math mode}
Math expressions are separate from text in \LaTeX.  To enter a math
environment in the middle of text, use the dollar sign \verb1$1, for
example \verb1$F = ma$1 produces $F = ma$.  Everything between the two
\verb1$1 signs will be treated as a math formula.

To type a math expression that is on its own line and centered, delimit
it with \verb1\[1 and \verb1\]1:

\begin{tabular}{ll}
Source: & Output: \\
\begin{minipage}{.5 \linewidth}
\begin{verbatim}
The following is an important equation:
\[ E = mc^2 \]
\end{verbatim}
\end{minipage}
&
\framebox{\begin{minipage}{.5 \linewidth}
The following is an important equation:
\[ E = mc^2 \]
\end{minipage}}
\end{tabular}

\bigskip

To give an equation a number and have it referable, use the
\verb1equation1 environment and use a \verb1\label1 command:

\begin{tabular}{ll}
Source: & Output: \\
\begin{minipage}{.5 \linewidth}
\begin{verbatim}
The following is an important equation:
\begin{equation}
\label{emc}
E = mc^2
\end{equation}
Please memorize Equation \ref{emc}.
\end{verbatim}
\end{minipage}
&
\framebox{\begin{minipage}{.5 \linewidth}
The following is an important equation:
\begin{equation}
\label{emc}
E = mc^2
\end{equation}
Please memorize Equation \ref{emc}.
\end{minipage}}
\end{tabular}

\bigskip

To typeset several equations together and have them properly aligned,
use the \verb1align1 environment:

\begin{tabular}{ll}
Source: & Output: \\
\begin{minipage}{.5 \linewidth}
\begin{verbatim}
Some important equations:
\begin{align}
\label{einstein}
E & = mc^2 \\
\label{newton}
F & = ma \\
\label{euler}
e^{i \pi} & = -1 
\end{align}
\end{verbatim}
\end{minipage}
&
\framebox{\begin{minipage}{.5 \linewidth}
Some important equations:
\begin{align}
\label{einstein}
E & = mc^2 \\
\label{newton}
F & = ma \\
\label{euler}
e^{i \pi} & = -1 
\end{align}
\end{minipage}}
\end{tabular}
The equations are aligned along the \verb1&1 and each line is
terminated by \verb1\\1.  To suppress the equation numbering (i.e. if
the equations won't be referred to) use \verb1align*1 instead of
\verb1align1.

\subsection{Writing math expressions}
I will only go over a few common mistakes and hard-to-find expressions
regarding how to write math expressions.  It is quite intuitive
otherwise, you can figure most things out quickly with trial and
error.  All expressions in math mode may be nested within each other
arbitrarily.

\begin{itemize}
\item Superscript and subscript are done using \verb1^1 and \verb1_1
characters.  Note that if you want multiple characters in the
super/subscript then you need to surround them with curly braces:
\verb2$e^i\pi = -1$2 gives $e^i\pi = -1$ whereas
\verb2$e^{i\pi} = -1$2 gives $e^{i\pi} = -1$.

\item Fractions are done using \verb3$\frac{1}{2}$3 which gives
  $\frac{1}{2}$.
\item To do a binomial coefficient, use \verb1$\binom{n}{k}$1 which
  gives $\binom{n}{k}$.
\item Modular arithmetic can be written using the \verb1\pmod{n}1 and
  \verb1\bmod{n}1 commands.  The first puts parentheses and a lot of
  space around the \texttt{mod} and the second does not.
\item $\forall$ and $\exists$ are written as \verb1\forall1 and
  \verb1\exists1.
\item $\neq$, $\geq$, and $\leq$ are written as \verb1\neq1, \verb1\geq1, and
  \verb1\leq1.
\item $\cdot$ (e.g. for multiplication) is \verb1\cdot1.
\item $\ldots$ is produced by \verb1\ldots1 and $\cdots$ by \verb1\cdots1 (notice the relative height).
\item $\circ$ is \verb1\circ1.
\item $\cup$, $\cap$, and $\setminus$ are \verb1\cup1, \verb1\cap1, and \verb1\setminus1.
% \item $\oplus$ is written with \verb1\oplus1.
\item Large $\cup$ and $\cap$ signs that behave like summations
  (see below for summations) are written as \verb1\bigcup1 and  \verb1\bigcap1.
\item $\Z$, $\R$, etc. are produced using \verb1\Z1, \verb1\R1, etc.
\item $\Exp$ is produced with \verb1\Exp1.
\item $\BigO{n}$, $\BigOmega{n}$, $\BigTheta{n}$ are produced with \verb1\BigO{n}1, \verb1\BigOmega{n}1, and \verb1\BigTheta{n}1.
\item $\P$ and $\NP$ are produced using \verb1\P1, \verb1\NP1, etc.
% \item $\calP$ is \verb1\calP1.
\item $\ell$ (as opposed to $l$) is produced with \verb1\ell1.
\item $\approx$ is produced with \verb1\approx1.
\item $\hat{x}$ and $\bar{x}$ are done with \verb1\hat{x}1 and
  \verb1\bar{x}1.  A longer bar may be written using
  \verb1\overline{\SAT}1, which produces $\overline{\SAT}$.
\item $x'$ and $x''$ are done with \verb1x'1 and \verb1x''1, respectively.
\item $\eps$ (as opposed to $\epsilon$) may be written with
  \verb1\eps1.
\item $\in$ and $\notin$ are written as \verb1\in1 and \verb1\notin1.
\item Negations may be done with \verb1\not1, for example
  \verb1\not\geq1 gives $\not\geq$.
\item $\zo$ is abbreviated as \verb1\zo1.
\item The probability sign $\Pr$ is defined as \verb1\Pr1.
\item $\{\}$ are done with \verb1\{1 and \verb1\}1.
\item To draw parentheses, braces, or brackets that resize to match the contents, use
  \verb1\left1 to precede the left one and \verb1\right1 to
  precede the right one:\\
Source:
\begin{verbatim}
\[ \Pr\left[\sum_{i=1}^k X_i > c \right] \leq 2^{-\Omega(c^2 k)} \]
\end{verbatim}
Output:
\[ \Pr\left[\sum_{i=1}^k X_i > c \right] \leq 2^{-\Omega(c^2 k)} \]

\item Arrays are similar to tables, except they must be used in place of
  tables when in math mode: instead of using \verb1\begin{tabular}1
  and \verb1\end{tabular}1 use \verb1\begin{array}1 and
  \verb1\end{array}1.  Again, you must give a column specification for
  how the columns are to be laid out.

\item Spacing is very different in math mode so text in the middle of
a formula is set strangely.  If you want to have text in the middle of
the formula, use the \verb1\text{some text}1 command.  For example,\\
\verb1$\P \neq \NP \text{ implies that } \SAT \notin \P$1 produces $\P
\neq \NP \text{ implies that } \SAT \notin \P$.

\item Summations and products are done using \verb1\sum1 and
  \verb1\prod1 respectively.  Parameters can be given for the
  summation/product as well:

\begin{tabular}{ll}
Source: & Output: \\
\begin{minipage}{.5 \linewidth}
\begin{verbatim}
\[ \sum_{i=1}^\infty \frac{1}{2^i} = 1 \]
\end{verbatim}
\end{minipage}
&
\framebox{\begin{minipage}{.5 \linewidth}
\[ \sum_{i=1}^\infty \frac{1}{2^i} = 1 \]
\end{minipage}}
\end{tabular}

\item Piecewise functions may be defined using the \verb1piecewise1
  environment:

\begin{tabular}{ll}
Source: & Output: \\
\begin{minipage}{.5 \linewidth}
\begin{verbatim}
\[ f(x) = \begin{piecewise}
1 & \text{ if } x = 0 \\
0 & \text{ if } x \neq 0
\end{piecewise} \]
\end{verbatim}
\end{minipage}
&
\framebox{\begin{minipage}{.5 \linewidth}
\[ f(x) = \begin{piecewise}
1 & \text{ if } x = 0 \\
0 & \text{ if } x \neq 0
\end{piecewise} \]
\end{minipage}}
\end{tabular}

\item You may define new commands using
  \verb1\newcommand{\commandname}{definition}1.  This is essentially
  a macro, so that whenever \verb1\commandname1 appears, the text of
  \verb1definition1 is inserted in its place.  If \verb1\commandname1
  is already taken, either use a different name or use
  \verb1\renewcommand{...}{...}1 to overwrite the old definition.
\item There are many many other symbols available.  You can search for
  ``latex symbols'' online and come up with the references.
\end{itemize}

\section{Writing pseudocode}
It is often necessary to state pseudocode when dealing with specific
algorithms.  Two packages that are useful for writing pseudocode are 
the \emph{algorithm} and \emph{algorithmcx} packages.
We use the \emph{algorithm} environment so that the code is not
across different pages.
We use the \emph{algorithmic} package to write the specific pseudocode.
An example is presented below.


\begin{tabular}{ll}
Source: & Output: \\
\begin{minipage}{.5 \linewidth}
\begin{verbatim}
\begin{algorithm}[H]
  \caption{Euclid's algorithm}
  \begin{algorithmic}
    \INPUT $a \ge 0$, $b \ge 0$
    \OUTPUT the gcd of $a$ and $b$
    \Function{gcd}{$a, b$}
      \If {$b = 0$}
        \Return $a$
      \Else $\;$
        \Return \Call{gcd}{$b, a \bmod b$}
      \EndIf
    \EndFunction
  \end{algorithmic}
\end{algorithm}
\end{verbatim}
\end{minipage}
&
\framebox{\begin{minipage}{.5 \linewidth}
\begin{algorithm}[H]
  \caption{Euclid's algorithm}
  \begin{algorithmic}
    \INPUT nonnegative integers $a$ and $b$
    \OUTPUT the greatest common divisor of $a$ and $b$
    \Function{gcd}{$a, b$}
      \If {$b = 0$}
        \Return $a$
      \Else $\;$
        \Return \Call{gcd}{$b, a \bmod b$}
      \EndIf
    \EndFunction
  \end{algorithmic}
\end{algorithm}
\end{minipage}}
\end{tabular}





\chardef\_=`\_
For a more comprehensive review (if-statements, for-loops, while-loops, etc), check out \\
\texttt{http://en.wikibooks.org/wiki/LaTeX/Algorithms\#Typesetting\_using\_the\_algorithmicx\_package}
and/or \texttt{http://get-software.net/macros/latex/contrib/algorithmicx/algorithmicx.pdf}.

\section{Figures}
The easiest way to include figures in \LaTeX{} is to draw the figure using some other software
package (such as Powerpoint, Keynote, Illustrator, xfig, or Google drawings) and save it as
a .pdf file or an image file (perhaps after cropping it to the desired size).  We use the
\emph{graphicx} package to import it into our document.
Suppose for example, we have an image pdf file saved in our working directory as 
``ClarkSnow.jpg" %``einstein.pdf''. 
We can use the following code to include the image in our document.


\begin{verbatim}
\begin{figure}[bht]
  \centering
  \includegraphics[width=0.95\textwidth]{ClarkSnow.jpg}
  \caption{Clark in Snow}
  \label{fig:snow}
\end{figure}
\end{verbatim}

\begin{figure}[bht]
  \centering
  \includegraphics[width=0.80\textwidth]{ClarkSnow.jpg}
  \caption{Clark in Snow}
  \label{fig:snow}
\end{figure}


For more details about figure formatting, check out \\
\texttt{http://en.wikibooks.org/wiki/LaTeX/Importing\_Graphics}.

\end{document}
